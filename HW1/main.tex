\documentclass{article} 
\usepackage{pgffor}
\usepackage{enumitem} 
\usepackage{multicol}
\usepackage{textgreek}
\makeatletter
\newcommand{\greekalpha}[1]{\c@greekalpha{#1}}
\newcommand{\c@greekalpha}[1]{%
  {%
    \ifcase\number\value{#1} %
    \or
    \textalpha
    \or
    \textbeta
    \or
    \textgamma
    \or
    \textdelta
    \or
    \textepsilon
    \or
    \textzeta
    \or
    \texteta
    \or
    \texttheta % or \straighttheta
    \or
    \textiota
    \or
    \textkappa
    \or
    \textlambda
    \or
    \textmu
    \or
    \textnu
    \or
    \textxi
    \or
    \textomikron
    \or
    \textrho
    \or
    \textpi
    \or
    \textsigma
    \or
    \texttau
    \or
    \textupsilon
    \or
    \textphi
    \or
    \textchi
    \or
    \textpsi
    \or
    \textomega
    \fi
  }%
}


\newcommand{\Greekalpha}[1]{\c@Greekalpha{#1}}
\newcommand{\c@Greekalpha}[1]{%
  {%
    \ifcase\number\value{#1} %
    \or
    \textAlpha
    \or
    \textBeta
    \or
    \textGamma
    \or
    \textDelta
    \or
    \textEpsilon
    \or
    \textZeta
    \or
    \textEta
    \or
    \textTheta
    \or
    \textIota
    \or
    \textKappa
    \or
    \textLambda
    \or
    \textMu
    \or
    \textNu
    \or
    \textXi
    \or
    \textOmikron
    \or
    \textRho
    \or
    \textPi
    \or
    \textSigma
    \or
    \textTau
    \or
    \textUpsilon
    \or
    \textPhi
    \or
    \textChi
    \or
    \textPsi
    \or
    \textOmega
    \fi
  }%
}



\AddEnumerateCounter*{\greekalpha}{\c@greekalpha}{5}

\AddEnumerateCounter*{\Greekalpha}{\c@Greekalpha}{5}
\makeatother
\usepackage[utf8]{inputenc}
\usepackage{amsmath}
\usepackage{enumitem}
\usepackage{systeme}
\usepackage{amssymb}
\usepackage{tikz} 
\usepackage{amsmath,amssymb,amsthm,amssymb}
\usepackage{physics}
\usepackage{arydshln}
\newcommand{\R}{\mathbb{R}}
\newcommand{\Pp}{\mathbb{P}}
\usepackage{datetime}
\newdateformat{mydate}{\shortmonthname[\the\month]  \the\year}

\usepackage{graphicx}

\makeatletter
\newcommand*\bigcdot{\mathpalette\bigcdot@{.5}}
\newcommand*\bigcdot@[2]{\mathbin{\vcenter{\hbox{\scalebox{#2}{$\m@th#1\bullet$}}}}}
\makeatother


\title{QBio II Homework 1}
\author{Shuai Yan}
\date{\today}
\begin{document}

\maketitle


%I use the Sator Square as my seal, please use something else. But this does work to show how the \array command works. 

% \[\begin{array}{ccccc}
% \text{R} & \text{O} & \text{T} & \text{A} & \text{S}\\
% \text{O} & \text{P} & \text{E} & \text{R} & \text{A}\\
% \text{T} & \text{E} & \text{N} & \text{E} & \text{T}\\
% \text{A} & \text{R} & \text{E} & \text{P} & \text{O}\\
% \text{S} & \text{A} & \text{T} & \text{O} & \text{R}  \\
% \end{array}
% \]



\newpage
\section{Question 1}
Given:

\[
    A = 
    \begin{bmatrix}
        2 & 1\\
        1 & 2
    \end{bmatrix}
\]

Solve $det(\lambda I-A)=0$ to get eigenvalues and eigenvectors.

\[
    \lambda I-A= 
    \begin{bmatrix}
        \lambda-2 & -1\\
        -1 & \lambda-2
    \end{bmatrix}
\]

\[
    det(\lambda I-A)=(\lambda-2)^2-1=0
\]

\[
    \Rightarrow \lambda_1=3, \lambda_2=1
\]

With eigenvalues, we can calculate the eigenvectors by solving $(\lambda
I-A)\vb{v}=0$.

Substitue $\lambda_1=3$ into the equation:

\[
    \begin{bmatrix}
        1 & -1\\
        -1 & 1
    \end{bmatrix}
    \cdot \vb{v}=0
\]

\[
    \Rightarrow \vb{v}_{\lambda_1}=
    \begin{bmatrix}
        1\\
        1
    \end{bmatrix}
\]

Then, substitue $\lambda_2=1$ into the equation:

\[
    \begin{bmatrix}
        -1 & -1\\
        -1 & -1
    \end{bmatrix}
    \cdot \vb{v}=0
\]

\[
\Rightarrow \vb{v}_{\lambda_2}=
    \begin{bmatrix}
        -1\\
        1
    \end{bmatrix}
\]

\section{Question 2}
The eigenvalue matrix is:

\[
    \Lambda=
    \begin{bmatrix}
        3 & 0\\
        0 & 1
    \end{bmatrix}
\]

\section{Question 3}
The eigenvector matrix is:

\[
    X=
    \begin{bmatrix}
        1 & -1\\
        1 & 1
    \end{bmatrix}
\]

\section{Question 4}

\begin{eqnarray}
    X\Lambda X^{-1}
    &=&
    \begin{bmatrix}
        1 & -1\\
        1 & 1
    \end{bmatrix}
    \cdot
    \begin{bmatrix}
        3 & 0\\
        0 & 1
    \end{bmatrix}
    \cdot
    \begin{bmatrix}
        \frac{1}{2} & \frac{1}{2}\\[6pt]
        -\frac{1}{2} & \frac{1}{2}
    \end{bmatrix} \nonumber \\
    &=&
    \begin{bmatrix}
        3 & -1\\
        3 & 1
    \end{bmatrix}
    \cdot
    \begin{bmatrix}
        \frac{1}{2} & \frac{1}{2}\\[6pt]
        -\frac{1}{2} & \frac{1}{2}
    \end{bmatrix} \nonumber \\
    &=&
    \begin{bmatrix}
        2 & 1\\
        1 & 2
    \end{bmatrix} \nonumber \\
    &=&
    A \nonumber
\end{eqnarray}
\textbf{\textit{Q.E.D}}

\section{Question 5}

\[
    A^6=(X\Lambda X^{-1})^6=X\Lambda^6X^{-1}
    =
    \begin{bmatrix}
        1 & -1\\
        1 & 1
    \end{bmatrix}
    \cdot
    \begin{bmatrix}
        729 & 0\\
        0 & 1
    \end{bmatrix}
    \cdot
    \begin{bmatrix}
        \frac{1}{2} & \frac{1}{2}\\[6pt]
        -\frac{1}{2} & \frac{1}{2}
    \end{bmatrix}
    =
    \begin{bmatrix}
        365 & 364\\
        364 & 365
    \end{bmatrix}
\]

\section{Question 6}
It will explode since $\lambda_1=3>1$.


% \newpage
% % You "section counter" should be set to one behind whatever problem you are on, such that the \section command gives what problem in the chapter you are working on. 
% \setcounter{section}{0}
% \section{Chapter.Section.Problem}
% \subsection{Goal}
% \subsection{Given}
% %%%%%%%%%%%%%%%%%%%%%%%%%%%%%%%%%
% %   below is an example of how to write a differential eq


% $$\frac{\delta y}{\delta x}= P(x)y=Q(x) $$

% %%%%%%%%%%%%%%%%%%%%%%%%%%%%%%%%%

% \subsection{Therefore}
% \textbf{\textit{Q.E.D}}


%if you have multiple sections in your p-set then use the below command to separate them again sticking to the Chapter.Section format 

% \newpage

% \begin{center}
% \begin{frame}{}
%   \centering \Huge
%   \emph{Chapter.Section}
% \end{frame}    
% \end{center}

\end{document}

